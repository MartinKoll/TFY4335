\section{02Method}

\textbf{2 Method}

\subsection{Apparatus}
\textbf{2.1.1 Apparatus}
Apparatus used in the ISO7 section in NanoLab at NTNU. The apparatus used in the experiment are listed below:

\begin{itemize}
    \item Optical microscope with camera by Zeiss Primo Star
    \item Dark field ring filter
    \item 10 {\micro\metre} micropipette
    \item C-Chip Hemocytometer
    \item White paint samples, for sample A and sample B
    \item Erioglaucine (food dye)
    \item Microfluidic chip, custom made using soft lithography
    \item Deionized water
    \item Syringe pump
    \item Syringes
    \item Microtubes
\end{itemize}

\subsection{Softwares}
\textbf{2.1.2 Softwares}
Softwares used in data capture and data processing:
\being{itemize}
    \item Zen software
    \item ImageJ
    \item TrackPy
\end{itemize}

\subsection{Experimental Procedure}
\textbf{2.2 Experimental Procedure}

\textbf{2.2.1 Optical microscope calibration}
\begin{enumerate}
    \item Adjust the optical microscope using Köhler illumination principle for optimal contrast and clarity.
    \item Calibrate the microscope according to the Zeiss Primo Star user guidelines.
\end{enumerate}

\textbf{2.2.2 Tracking particles to estimate Brownian motion and diffusion constant:}
\begin{itemize}
    \item Prepare a small sample of white paint containing microparticles (latex and nanoparticles) by spreading a thin layer on a glass slide.
    \item Place the sample under the optical microscope.
\end{itemize}

\subsection{Processing of data}
