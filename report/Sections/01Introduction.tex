\section{Introduction}


Diffusion is the process of net movement of molecules in a solution from areas of high concentration
to areas of low concentration. This movement is driven
by the random thermal motion of molecules, known as Brownian motion \cite[512]{author2020book}.
Brownian motion is the erratic and random motion of particles suspended in a fluid,
resulting from their collision with fast-moving molecules in the fluid.
Diffusion is a fundamental mechanism in many biological processes,
such as mass transport in and around cells,
distribution of signaling molecules, and mixing of cellular components.
Understanding diffusion and Brownian motion is therefore crucial for understanding biological systems.

The main goal of this exercise is to study diffusion of particles with size distributions in the $nm$ - $\mu m$ range.
This is done using two different methods. The first method leverages modern camera technology and image processing to track the movement of
paint particles which are observable in a light microscope (particles in the $\mu m$ size range).
By calculating the mean squared displacement
of the particles, the diffusion coefficient and hydrodynamic radius of the particles can be estimated.
This method is largely similar to the method used by Perrin in 1908 to determine Avogadro's number (sitering Newburgh her).
The second method utilizes microfluidics to create a concentration gradient of diffusing particles which
cannot be observed directly in a microscope. By measuring the concentration of the particles at different points in the microfluidic channel,
the diffusion coefficient and hydrodynamic radius can be estimated without directly observing the particles (kilde diffusion-based extraction).


\subsection{Theory of diffusion}

